\documentclass[11pt]{article}

\usepackage[paperwidth=2.75in,paperheight=3.75in,width=2.25in,height=3.25in]{geometry}

\pagestyle{empty}
\thispagestyle{empty}

\tolerance=1
\emergencystretch=\maxdimen
\hyphenpenalty=100
\hbadness=10000

\newcommand\myssection[1]{\begin{center}\bf\small\noindent #1\end{center}}
\newcommand\myheader[1]{\begin{center}\small\sc #1\end{center}}

\begin{document}

\myheader{\Large\sc Absolute Zero}

\tiny

\begin{center}
  \em 2-5 players\\
  age 5+
\end{center}

\noindent Absolute Zero plays as a standard trump game, where suits
are defined by color and the white suit is trump.  Deal rotates
clockwise, and the dealer leads the first trick.

Players must follow suit if they have a card with the same color,
otherwise they may play any card.  The winner of each trick is the
player who played the highest-value white card (keeping in mind that
positive numbers are greater than negative values), or the highest
card in the color that was lead if no white card was played.  The player
who wins a trick leads for the next trick.

\myssection{Scoring} At the end of each hand, players count the
total value of tricks they captured.  If this number is nonzero, their
score for the hand is the absolute value of this integer.  If this
number is zero, then the player's score for the hand is the total
number of cards captured.

Play continues until one player reaches a cumulate score of 100.

\myssection{Cooperative play} As an alternative, play can be
cooperative:  the total score for each hand is the total of all
players contributions for that hand.  Play continues until the total
reaches 100 times the number of players.

\end{document}
