\documentclass[11pt]{article}

\usepackage[paperwidth=2.75in,paperheight=3.75in,width=2.25in,height=3.25in]{geometry}

\pagestyle{empty}
\thispagestyle{empty}

\tolerance=1
\emergencystretch=\maxdimen
\hyphenpenalty=100
\hbadness=10000

\newcommand\myssection[1]{\begin{center}\bf\small\noindent #1\end{center}}
\newcommand\myheader[1]{\begin{center}\small\sc #1\end{center}}

\begin{document}

\myheader{\Large\sc Physics! \footnotesize \\  the game}

\tiny

\myssection{Setup} Begin by dealing out all the cards as face-down
draw piles for each player.  Each player draws $N$ cards.

\myssection{Play} Play begins with the youngest player and proceeds
clockwise.  Each player has a choice to
\begin{enumerate}
\item Declare themselves the winner, if their hand and draw pile are
  empty.
\item Play a card face-up into the play area.  The card may be played
  onto one or more existing cards, indicating that it modifies those
  cards.  The player then draws a card from their draw pile if
  possible.
\item Challenge the player who previously played.
\end{enumerate}

\clearpage
\myheader{Challenge!} The challenged player must describe a tenable
single-part physics problem that incorporates all the elements in
play.  Each element must have relevance to the outcome of the problem,
and the problem must be solvable.  The players must reach consensus as
to whether this has been achieved, possibly with the challenged player
actually solving the problem.

\myssection{Problem solved} The cards in play are moved to the discard
pile, and play continues with the challenged player smugly taking the
next turn.

\myssection{No adequate problem} The challenged player shuffles the
cards in play into his draw pile with chagrin, and play continues with
the challenger gleefully taking the next turn.

\end{document}
