\documentclass[11pt]{article}

\usepackage[paperwidth=2.75in,paperheight=3.75in,width=2.0in,height=3.0in]{geometry}
\usepackage{tikz}

\pagestyle{empty}
\thispagestyle{empty}

\tolerance=1
\emergencystretch=\maxdimen
\hyphenpenalty=100
\hbadness=10000

\newcommand\myssection[1]{\begin{center}\bf\small\noindent #1\end{center}}
\newcommand\myheader[1]{\begin{center}\small\sc #1\end{center}}

\begin{document}

\color{black}
\begin{tikzpicture}[remember picture,overlay]
  \node [shape=rectangle, fill=black, minimum height=\paperheight, minimum width=\paperwidth, anchor=south west] at (current page.south west) {};
  \node [shape=circle, fill=white, minimum width=1.3in] at (current page.center) {};
  \node [shape=circle, fill=red, minimum width=.75in] at ([xshift=0in, yshift=.75in]current page.center) {};
  \node [shape=circle, fill=red, minimum width=.75in] at ([xshift=0in, yshift=-.75in]current page.center) {};
  \node [shape=circle, fill=blue, minimum width=.75in] at ([xshift=.65in, yshift=.375in]current page.center) {};
  \node [shape=circle, fill=blue, minimum width=.75in] at ([xshift=-.65in, yshift=-.375in]current page.center) {};
  \node [shape=circle, fill=green, minimum width=.75in] at ([xshift=.65in, yshift=-.375in]current page.center) {};
  \node [shape=circle, fill=green, minimum width=.75in] at ([xshift=-.65in, yshift=.375in]current page.center) {};
  \node [shape=rectangle, fill=white, minimum width=.45in, minimum height=.15in] at ([xshift=-.65in, yshift=.375in]current page.center) {};
  \node [shape=rectangle, fill=white, minimum width=.45in, minimum height=.15in] at ([xshift=+.65in, yshift=-.375in]current page.center) {};
  \node [shape=rectangle, fill=white, minimum width=.45in, minimum height=.15in] at ([xshift=+.65in, yshift=.375in]current page.center) {};
  \node [shape=rectangle, fill=white, minimum width=.45in, minimum height=.15in] at ([xshift=-.65in, yshift=-.375in]current page.center) {};
  \node [shape=rectangle, fill=white, minimum width=.45in, minimum height=.15in] at ([xshift=0in, yshift=.75in]current page.center) {};
  \node [shape=rectangle, fill=white, minimum width=.45in, minimum height=.15in] at ([xshift=0in, yshift=-.75in]current page.center) {};
  \node [shape=rectangle, fill=white, minimum width=.15in, minimum height=.45in] at ([xshift=-.65in, yshift=.375in]current page.center) {};
  \node [shape=rectangle, fill=white, minimum width=.15in, minimum height=.45in] at ([xshift=+.65in, yshift=.375in]current page.center) {};
  \node [shape=rectangle, fill=white, minimum width=.15in, minimum height=.45in] at ([yshift=-.75in]current page.center) {};
  \node [anchor=north] at ([yshift=-.4in] current page.north) {\color{white}\Large\sc Absolute Zero};
\end{tikzpicture}
\clearpage

\begin{tikzpicture}[remember picture,overlay]
    \node [shape=rectangle, fill=red, minimum height=\paperheight,
      minimum width=\paperwidth, anchor=south west] at (current
    page.south west) {};
    \node [shape=rectangle, fill=white, minimum height=3.25in,
      minimum width=2.25in, anchor=center] at (current
    page.center) {};
\end{tikzpicture}
%\color{white}
\footnotesize

\noindent {\sc Absolute Zero} plays as a standard trump game, where suits
are defined by color and the white suit is trump.  Deal rotates
clockwise, and the dealer leads the first trick.

Players must follow suit if they have a card with the same color,
otherwise they may play any card.  The winner of each trick is the
player who played the highest-value white card (keeping in mind that
positive values are greater than negative values), or the highest
card in the color that was lead if no white card was played.  The player
who wins a trick leads for the next trick.

\clearpage
\begin{tikzpicture}[remember picture,overlay]
    \node [shape=rectangle, fill=blue, minimum height=\paperheight,
      minimum width=\paperwidth, anchor=south west] at (current
    page.south west) {};
    \node [shape=rectangle, fill=white, minimum height=3.25in,
      minimum width=2.25in, anchor=center] at (current
    page.center) {};
\end{tikzpicture}

\myssection{Scoring} At the end of each hand, players count the
total value of tricks they captured.  If this number is nonzero, their
score for the hand is the absolute value of this integer.  If this
number is zero, then the player's score for the hand is the total
number of cards that player captured.

Play continues until one player reaches a cumulative score of 100.  If
more than one exceeds 100 after a hand, then the player with the
highest score wins.

\clearpage
\begin{tikzpicture}[remember picture,overlay]
    \node [shape=rectangle, fill=green, minimum height=\paperheight,
      minimum width=\paperwidth, anchor=south west] at (current
    page.south west) {};
    \node [shape=rectangle, fill=white, minimum height=3.25in,
      minimum width=2.25in, anchor=center] at (current
    page.center) {};
\end{tikzpicture}

\vspace{-0.1in}
\begin{center}
  \em 2-5 players\\
  age 5+
\end{center}

\noindent {\sc Absolute Zero} is simple enough to be learned by young
children, while still being challenging for adults.

\myssection{Cooperative play} Play can be cooperative: the total score
for each hand is the total of all players contributions for that hand.
Play continues until the total reaches 100 times the number of
players.  This is particularly suitable when there is a wide disparity
in age or skill.

\end{document}
